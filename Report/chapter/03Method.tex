\section{Method}
To compare the various learning algorithms, we have written a multi-agent simulation in Python in which agents have to move an object to a certain goal area. In this section, the various components of the simulation will be explained.
\subsection{The Environment}
The environment consists of an \textbf{x by y} grid in which the agents can move around. There are a number of different types of cells:
\paragraph{Free cells}
These are cells that agents can freely move to. Once an agent or a block moves to a free cell, this cell becomes occupied. They are denoted by dots in the visual representation of the simulation.
\paragraph{Walls}
These cells are occupied, so an agent can never move to them. They are initialized at the start of the simulation and will not change during the simulation. There are outer walls, which are on the edges of the grid, and some walls inside the grid to make the environment more complex. They look like hash tags in the visualization.
\paragraph{Block}
The block is an item that has to be transported to a certain goal area by the agents. It occupies one cell, and agents cannot move through it. It can only be moved if it is grasped by all of the agents present in the simulation, and those agents move in the same direction. If the block reaches the goal, an epoch ends. In the simulation, it is denoted by a B.
\paragraph{Goal}
The goal cell is basically the same as a free cell, with the exception that if the block reaches the goal a reward is given to the agents and the run of the simulation ends. Agents can freely move on and over the goal, just like with free cells. The goal is represented by a B in the simulation.
\paragraph{Agent}
We'll talk about these in more detail in the next subsection. They occupy one cell, so other agents cannot move through each other. In the simulation they are represented by A's.
In \textbf{FIGURE X} you can see how the visual representation of the simulation looks.
\subsection{The Agents}
For our initial simulation, we start out with two agents. At the start of a run, the agents are spawned at their respective start locations which are defined in the algorithm. Each step, an agent can perform one of five actions: \textbf{Stay}, \textbf{Left}, \textbf{Right}, \textbf{Up}, \textbf{Down} or \textbf{Grab}. The actions mostly speak for themselves. The agent can stay where it is, move in one of four directions or grab the block. The \textit{grab} action can be performed at any step, but will only do something when an agent is next to the block. Once an agent has grabbed the block it will not let go. The move actions \textit{left, up, right} and \textit{down} only do something when the cell the agent wants to move to is actually free.

Each step, an agent chooses the best action based on its learning algorithm. Then, the agents perform their actions and update the values related to their learning algorithms (usually these are the values in their Q-tables). This process repeats itself until the block has reached the goal. After that, a new run starts and the agents are reset to their starting positions. There are two ways for an agent to get a reward: they can either grab the block to get a reward of 10, or they can get the block to the goal for a reward of 100. These rewards are used to calculate the Q-values, which we will talk about more in the next section.

\subsection{The Learning Algorithms}
For this paper, we compared two machine learning algorithms: Q-learning and team Q-learning. Both are reinforcement learning algorithms, and they are based on the same principles. The goal of both algorithms is, in our case, to make sure that the agents find the optimal policy that will result in the highest reward. It can be found through a value iteration method.

\textit{Q-learning} was first introduced by Watkins in 1989 \cite{watkins1989}. It is a model free reinforcement learning technique for multi-agent systems. In our simulation, the agents begin with a Q-table filled with zeros. Each step, they choose one of the five actions with certain probabilities, which are based on the different Q-values. The probabilities are calculated as follows: $P(a_{k})=\frac{e^{Q(s,a_{k})/ \tau}}{\sum_{l=1}^{m}e^{Q(s,a_{l})/ \tau}}$ In this case, $s$ is the current state, $m$ is the total number of Q-values for the current state, $a_{k}$ is the $k^{th}$ action and $\tau$ is a parameter that gradually decreases during the simulation. After an action is chosen and a new state $s'$ is reached, the value in $Q(a_{k})$ is updated as follows: $Q(s,a_{k}) = (1- \varepsilon) Q(s,a_{k})+ \varepsilon (r+ \beta maxQ[s',a'])$ (where $\varepsilon$ is the learning rate between 0 and 1). From this formula, it becomes apparent that in order to calculate the Q-value for a certain action in a certain state, both the outcome of that action and the outcome of the best possible action in the next state are considered. After each epoch, to make sure the algorithm converges, we update $\tau$: $\tau_{i} = \tau_{i}-(\tau_{1}-0.1/N)$, where $\tau_{i}$ is the value of $\tau$-value in epoch i and $N$ is the total number of epochs.

After the agents have reached the goal with the block, their Q-tables are transferred to the next run. This should result in the fact that the agents learn to choose the path that gains them the highest reward. One notable thing in our simulation is that the agents have two separate Q-tables: one for when the agents have grasped the block, and one for when they have not. This is necessary because the best actions change when the agents start moving the block. Thus, they have to optimize two seperate paths.

\textit{Team Q-learning} is based on the same principles as Q-learning, and is actually an extension of the Nash Q-learning algorithm. The primary difference is that in team Q-learning, the agents share a single Q-table, and the Q-values are based on their joint actions. In this case, each step the best action pair is chosen based on the current position of both of the agents. So, after both agents have taken an action, the Q-value for the joint position and joint action of the agents is still updated with the same formula, but in this case,  $Q(s,a_{k})$ is the Q-value for the shared action $a_{k}$ in the combined state $s$. The probability that a certain action is taken is calculated in a way that is very similar to Q-learning. The main difference is that the probabilities for shared actions are calculated, instead of probabilities for the actions of the individual agents. So, in team Q-learning the agents share a Q-table and action probabilities. With this algorithm, it is like the agents try to cooperate with each other.

\subsection{The simulation}
The previous subsections explained how the individual parts of our simulation work. However, there is still an question that needs to be answered: how do all these pieces fit together to form a coherent simulation? There are two different algorithms, one for Q-learning and one for team Q-learning. Although they are quite similar, there are some significant differences. In this subsection, the basis of the algorithms will be explained, and the differences will be highlighted.\\
The program is executed with two arguments: one for the way in which the actions are chosen (because we can also use the epsilon-greedy strategy, although we do not use it for this paper) and one for the environment. The environment can either be simple, in which case it is just a five by five grid without inner walls, or complex. The complex world is the world that can be seen in \textbf{FIGURE X}. The outer loop of the algorithms loops through the number of runs, which is usually one. In it, $\tau$ and a few other parameters are initialized. Then, the algorithms loop through the right number of epochs, which are divided in a training and a test part. Each epoch begins by resetting the world and putting the agents back in their starting positions. Then, it goes into a loop that runs until the goal is reached. Each step, the agents choose an action based on the Q-values for the state they are in. In team Q-learning, a set of actions is chosen, while in Q-learning, both of the agents have to individually choose an action. Although we do not use it for this paper, the actions can also be chosen with the epsilon-greedy strategy. Usually, an action (or a set of actions) is chosen with a probability relating to its Q-value, as we saw in the previous subsection. After the actions have been chosen, the world is updated. For this to be done succesfully, both of the algorithms have to check a few things: do the agents move in to a space that is occupied? Have the agents grasped the block? If they have grasped the block, are they moving in the same direction? All of these things have to be checked to see if the action can be succesfully performed or if one or more of the agents stay in their current state.